\documentclass[here]{article}
\usepackage[utf8]{inputenc}
\usepackage{booktabs,siunitx,tabularx, natbib}
\usepackage[dvipdfmx]{hyperref}
\usepackage{comment}
\usepackage{amsmath}

\bibliographystyle{jecon}
\bibpunct[:]{(}{)}{,}{a}{}{,}

\title{What makes Legislative Attention}
\author{Yuki TSUJIMURA}
\date{Date}

\begin{document}
\maketitle

\begin{abstract}
アメリカ議会における立法生産性の議論は、議会が適切に機能しているのかという問題関心から多くの分析が蓄積されてきた。しかし、立法生産がアメリカ社会に応答的な形でなされてきたのかについては十分な検討がなされていない。本稿では、1984年から2013年までのデータを用いて、政策領域に対する社会的な注目度が、当該政策領域における立法生産性といかなる関係を示すのかを分析した。分析の結果、政策領域に対する注目度が高まることで、議会の生産性が低下し得ることが示された。加えて、こうした影響は議会の審議段階によっても異なることが観察された。
\end{abstract}


\section{はじめに}
- motivation: 
%アメリカ議会における立法生産性の議論は、議会が適切に機能しているのかという問題関心から多くの分析が蓄積されてきた。
\\
 アメリカ議会が十分に機能しているのかという問題は、議会に対する根本的な課題のひとつとして大きな関心の対象となってきた。\\
- main contribution: 
%しかし、立法生産がアメリカ社会に応答的な形でなされてきたのかについては十分な検討がなされていない。本稿では、1984年から2013年までのデータを用いて、政策領域に対する社会的な注目度が、当該政策領域における立法生産性といかなる関係を示すのかを分析した。
\\
 本稿では、議会が社会に応答しているのかという問題を立法生産性の観点から検討し、社会的に関心の高まっている政策領域では、逆に立法が停滞するという関係が観察されることを指摘する。\\
 近年の立法生産性(legislative productivity)にかかわる議論はDavid Mayhewの\textit{Divided We Govern}を転機として様々な議論が蓄積されてきた。\\
 Sarah Binderは立法生産性の議論に、「議会が社会の問題に適切に対応しているのか」という応答性の問題から再検討を加え、社会的な注目度が議会における膠着状態を緩和させる可能性を指摘した。\citep*{Binder2003-bn}\\
\\
- in detail discription:
%分析の結果、政策領域に対する注目度が高まることで、議会の生産性が低下し得ることが示された。加えて、こうした影響は議会の審議段階によっても異なることが観察された。
\\
 議会の立法生産性と社会的注目度との関係は、その重要性に比して分析が少ない\footnote{重要な例外として\citet*{Adler2013-ay}}。本稿では、政策領域への注目度と立法生産との関係を長期のデータから検討することで、従来の分析では析出されてこなかった関係を示す。\\
 加えて、先行研究では議会審議の程度を論じる際に、異なる法案審議の段階を用いた分析がなされてきた。本稿では、異なる審議段階における議会の応答性を検討することで、社会的な注目度が法案審議に対して与える影響は、どの段階の法案審議かによっても異なっていることを指摘する。\\

\section{先行研究}
\subsection{立法生産性の説明モデル}
 議会の立法生産性についての議論は、David Mayhewの\textit{Divided We Govern}を転機として、議論が蓄積されてきた。\citet*{Mayhew1991-rq}は、従来比較的広く論じられてきた命題である、「議会多数派と大統領の選出政党が異なる分割政府のもとでは、議会での法案審議は停滞する」という命題に対して、重要な法案についての議決数は分割政府下でも明確に低下しているわけではないことを示した。\\
 立法生産性と分割政府との関係はその後も多面的に検討がなされた。\\
 Sarah Binderは従来の立法生産性の議論が、単に議会が法案をどの程度審議してきたのかという数の問題に終始してきたことを問題視した。むしろ重要なのは社会的な課題を議会がどの程度解決できているかという点であり、課題のリストの中から議会がどの程度それらの課題に対して、法案提出という点で応答しているのかが重要であると論じた。\citep*{Binder2003-bn,Binder2017-wr}\\
 \citet*{Binder2003-bn}の議論の主眼は、立法のこう着状態を議会間の政策位置の違いなどに注目して説明する点にあり、政策領域への注目度の高さがこう着状態を緩和する傾向にあることを示唆しているに過ぎない。この点をさらに検討しているのが\citet*{Adler2013-ay}である。この議論では、政策領域への注目度が当該政策領域における立法生産を促進すると指摘している。\\

\subsection{「政策領域への注目度」とは何か}
 \citet*{Binder2003-bn}や\citet*{Adler2013-ay}の議論は、従来の議会におけるこう着状態をめぐる議論が注目してきた政治的変数だけではなく、社会的変数である有権者の関心や社会的な注目度といった要素に目を向けることの重要性を示すものである。\\
 政策領域に対する注目度を説明変数として政治現象を論じる枠組みは大まかには2つの議論が存在する。第一には、議員が選挙区の関心をいかに反映しているのかを主眼に据えた分析である。第二に挙げられるのは、社会的な関心の高さを、政策変化の契機として理解する議論である。\\
 第一の、議員と選挙区との関係を論じる議論では、議員は選挙区の関心の高さに応じて法案の提出、投票を行なっていると考える。\\
 第二の、政策変化の契機として注目度の高さを捉える分析では、政策の固着性を前提として政策変化が生じるためには、従来検討されて来なかった新しい側面が導入されることによる政治的均衡の変化が必要だと考える。\citep*{Baumgartner1993-bc}\\
 以上の議論はいずれにしても、政策領域に対する注目度や関心の高さを、議会のこう着状態を打破する変数として機能することを前提としている。\\
 他方で、政策領域に対する注目度は必ずしも政策変化や審議を推し進める変数として機能するとは限らない。\\
 すなわち、政策領域に対して関心が高まるという現象は、法案審議を停滞させる機能を果たす場合もあり得るということである。\\
 以下では、社会的な注目度と法案審議との関係性を観察データを用いて分析する。\\

\section{分析方法}
\subsection{データ}
 分析に際しては、Comparative Agendas Project(CAP)によって収集・整理されたデータセットを利用する。\\

\subsubsection{従属変数}
 本稿の従属変数は、政策領域ごとの立法生産性である。実際の立法生産性の計測方法は様々にあり得るが、本分析では政策領域ごとに審議された法案の数を採用する。\\
 CAPのデータセットには、議会に提出された法案をCAPの用いる政策領域ごとに分類したデータが公開されている。本分析では法案が提出された年に政策領域ごとの法案数を合算した値を従属変数として採用する。\\
 加えて、法案審議に見られる様々な傾向は法案の審議段階によっても異なると考えられる。アメリカ連邦議会には、例年数千件から一万件近い数の法案が提出されている。他方、決議が行われる法案はその内の1割程度である。最終的は両院を通過した法案に大統領が署名を行なってはじめて法案が法律となる。したがって、本稿では審議の各段階、具体的には\textcircled{\scriptsize 1}法案提出、\textcircled{\scriptsize 2}議決、\textcircled{\scriptsize 3}立法化の3段階についてそれぞれ政策領域ごとの法案数\footnote{本稿では法案数をBillとJoint Resolutionの合算値によって導出している。したがって、Simple ResolutinoとConccurent Resolutionは分析から除外されている。これは、立法化され得る議案が前者2種類のみであり、後者は制度上立法化できないためである。\citep*{US_Senate2020-xq}この操作化については、\citet*{Adler2013-ay}も同様の処理を行っっており、妥当であると思われる。}を集計し、それぞれを従属変数とする分析を行った。\\

\subsubsection{独立変数}
 本稿の独立変数は、政策領域に対する注目度である。本分析では、\citet*{Adler2013-ay}の議論を参考に2つの変数を用いて注目度を計測した。\\
\paragraph*{メディア報道量}
 第一に、メディアの報道量である。具体的にはNew York Timesの報道について、CAPが分類した政策領域ごとの報道量を計算している。\\
 本分析では、メディア報道量の計算式を修正して用いている。これは、CAPの提示する計測方式では異なる年度間の比較を行う上で不適切であると考えられるからである。CAPによるメディア報道量の計測方法は、New York Times Indexの奇数ページに記載された最初の記事をCAPの分類コードに従って分類している。コーディングされた各記事は、記事に付されている概要文の長さによって、長・中・短の3つに割り付けられており、単なる記事の本数ではなく、その重要性を加味することができる。\\
 このサンプリング方法は、2つの面で複数年度の比較に適さない特性を有する。第一に、New York Times Indexは年度によってその全ページ数が異なるため、各記事の長さについての情報は異なる年度で単純な比較ができない。第二に、年度毎にページレイアウトが若干異なるため、ページ毎の記事数も年度によって異なっている。\\
 年度ごとの政策領域毎の報道量を説明変数として用いる場合、最も簡単な操作化は、年度ごとの各政策領域における報道量を、記事の重みを加味して合算することである。この場合、政策領域$i$における報道量$MediaCoverage_i$は、記事の長さ$j = 1,2,3$と長さ$j$の記事数$entry_{ij}$によって表すことができる。\\
\begin{align}
  MediaCoverage_i =  \sum_{j=1}^3 entry_{ij}*j
\end{align}
しかし、上述した通り年度毎にNew York Times Indexのページ数やページ毎の記事数が異なるため、本来であれば、この両者を加味した形で操作化することが望ましい。したがって、本稿では、(1)式を年度$k$の全ページ数$NumOfPage_k$とページ毎記事数$PerPage_k$によって調整した値を報道量の計算んしきとして採用する。すなわち年度$k$における政策領域$i$の報道量は以下の式で表される。\\
\begin{align}
  MediaCoverage_{ik} = \frac{PerPage_k}{NumOfPage_k} \sum_{j=1}^3 entry_{ij}*j
\end{align}
端的に言えば本式は、(1)で導出される各記事の「長さ」を、その年のNew York Times Indexのページ数が多い場合にはより短く、その年の各ページに記載された記事が多い場合にはより長く評価した値である。

\paragraph*{有権者の関心度合い}
 政策領域に対する社会的な関心の度合いを示す第二の独立変数は有権者が各政策領域に対してどの程度関心を払っているのかという問題である。Gallup社は「今日この国が直面している最も重要な問題はなにか("the most important problem facing this country today.")」という質問を定期的に有権者に対し行っている。\citep*{Gallup_Inc2006-vs}\\
 CAPはこのGallupによる定期的なサーベイの質問票を利用し、CAPが用意している政策領域それぞれについて有権者がどの程度関心を寄せているのかを年度ごとのデータとして用意している。本分析では、Gallup MIPで各政策領域が最も重要な争点であると答えた有権者の比率を有権者の各政策領域に対する関心度合いを示す指標として用いる。\\

\subsubsection{統制変数}
 法案審議がどの程度行われるのかは、必ずしも社会的な要因によってのみ影響されるわけではない。特に独立変数との関係で統制すべきと判断される要素については統制変数として回帰式に投入した。\\
 第一に、分割政府ダミーである。分割政府と統一政府とでは、立法生産に対して異なる状況が観察されることが指摘されてきた。\\
 第二に、選挙年ダミーである。本分析では年度毎に提出された法案について集計した上で分析を行っている。中間選挙を含めた選挙年と非選挙年では提出される法案数に大きな乖離が存在する。そのため、連邦レベルの選挙が存在する年とそうでない年を区分するために選挙年ダミーを投入した。\\
 第三に、政策領域ダミーを投入し、政策領域ごとの扱われやすさを統制している。例えば、マクロ経済と運輸とでは例年提出される法案数も有権者やメディアでの報道数にも差がある。政策領域を統制することで、これら所与の差異による影響を低減させる必要がある。\\
 第四に、多数党ダミーを投入している。多数党が法案審議過程に対して影響力を行使していることはしばしば指摘される事実である。\citep*{Rohde1991-da,Aldrich1995-xf,Cox2005-pn,Cox2007-xq}したがって、法案審議の過程においても多数党によって提出された法案と、少数党によって提出された法案では異なる取り扱いがなされると考えることが妥当である。この点についてもダミー変数を投入して統制している。\\
\subsection{分析モデル}
 分析には単純な最小二乗法(OLS)を利用し、政策領域ごとの\textcircled{\scriptsize 1}法案提出数(上院/下院)、\textcircled{\scriptsize 2}議決に付された法案数(上院/下院)、\textcircled{\scriptsize 3}立法化された法案数を従属変数としたモデルについて上述の独立変数・統制変数を投入した分析を行った。\\

\section{分析結果}
ここに分析結果


\begin{table}[]
  \caption{House (Bill Introduction)}
  %\centering
  \hspace{-1cm}

  \begin{tabular}[t]{lccc}
    \toprule
      & NYT v. Bill & MIP v. Bill & NYT + MIP v. Bill\\
    \midrule
    NYT Coverage & \num{0.862} &  & \num{0.026}\\
    & (\num{0.701}) &  & (\num{0.991})\\
    Gallup MIP &  & \num{85.433}* & \num{85.371}*\\
    &  & (\num{0.015}) & (\num{0.016})\\
    Divided Government (dummy) & \num{-1.972} & \num{-1.832} & \num{-1.836}\\
    & (\num{0.567}) & (\num{0.592}) & (\num{0.593})\\
    Bills from Majority Party (dummy) & \num{22.324}*** & \num{22.324}*** & \num{22.324}***\\
    & (\num{0.000}) & (\num{0.000}) & \vphantom{1} (\num{0.000})\\
    Major Topic & Yes & Yes & Yes\\
    Election Year (dummy) & \num{-43.996}*** & \num{-44.021}*** & \num{-44.021}***\\
    & (\num{0.000}) & (\num{0.000}) & (\num{0.000})\\
    Num.Obs. & \num{841} & \num{841} & \num{841}\\
    \midrule
    AIC & \num{8878.7} & \num{8872.7} & \num{8874.7}\\
    BIC & \num{9001.8} & \num{8995.8} & \num{9002.5}\\
    Log.Lik. & \num{-4413.343} & \num{-4410.345} & \num{-4410.345}\\
    F & \num{36.541} & \num{37.045} & \num{35.520}\\
    RMSE & \num{46.71} & \num{46.54} & \num{46.57}\\
    \bottomrule
    \multicolumn{4}{l}{\rule{0pt}{1em}. p $<$ 0.1, * p $<$ 0.05, ** p $<$ 0.01, *** p $<$ 0.001}\\
  \end{tabular}
\end{table}

 まず、法案提出について検討する。表1は、政策領域ごとに下院に提出された法案数を独立変数と統制変数を投入して回帰式を推定した点推定値である。まず指摘できるのは、下院に提出された法案では有権者の関心度合い(Gallup MIP)が一貫して有意に正の推定値を示している点である。これは、議会に対して提出される法案全体で見た場合に、有権者が関心を持っている争点が議会に反映されやすいことを示しており、議員が選挙区民に応答的に法案提出を行っているという既存の知見と整合的な結果である。\citep*{Page1983-bx,Monroe1998-ty,Burstein2003-vs}\\

\begin{table}[htbp]

  \caption{Senate (Bill Introduction)}
  \centering
  \begin{tabular}[t]{lccc}
  \toprule
    & NYT v. Bill & MIP v. Bill & NYT + MIP v. Bill\\
  \midrule
  NYT Coverage & \num{-0.860} &  & \num{-1.382}\\
   & (\num{0.660}) &  & (\num{0.484})\\
  Gallup MIP &  & \num{49.936} & \num{53.232}.\\
   &  & (\num{0.101}) & (\num{0.084})\\
  Divided Government (dummy) & \num{-4.935}. & \num{-5.058}. & \num{-4.850}\\
   & (\num{0.100}) & (\num{0.090}) & (\num{0.105})\\
  Bills from Majority Party (dummy) & \num{18.918}*** & \num{18.919}*** & \num{18.918}***\\
   & (\num{0.000}) & (\num{0.000}) & \vphantom{1} (\num{0.000})\\
  Major Topic & Yes & Yes & Yes\\
  Election Year (dummy) & \num{-20.686}*** & \num{-20.697}*** & \num{-20.702}***\\
   & (\num{0.000}) & (\num{0.000}) & (\num{0.000})\\
  Num.Obs. & \num{839} & \num{839} & \num{839}\\
  \midrule
  AIC & \num{8622.2} & \num{8619.6} & \num{8621.1}\\
  BIC & \num{8745.3} & \num{8742.7} & \num{8748.9}\\
  Log.Lik. & \num{-4285.111} & \num{-4283.824} & \num{-4283.571}\\
  F & \num{17.323} & \num{17.480} & \num{16.790}\\
  RMSE & \num{40.59} & \num{40.53} & \num{40.54}\\
  \bottomrule
  \multicolumn{4}{l}{\rule{0pt}{1em}. p $<$ 0.1, * p $<$ 0.05, ** p $<$ 0.01, *** p $<$ 0.001}\\
  \end{tabular}

\end{table}



\section{含意}

\newpage
\bibliography{reference}
\end{document}