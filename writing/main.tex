\documentclass[here]{article}
\usepackage[utf8]{inputenc}
\usepackage{booktabs,siunitx,tabularx, natbib}
\usepackage[dvipdfmx]{hyperref}
\usepackage{comment}
\usepackage{amsmath}

\bibliographystyle{jecon}
\bibpunct[:]{(}{)}{,}{a}{}{,}

\title{What makes Legislative Attention}
\author{Yuki TSUJIMURA}
\date{Date}

\begin{document}
\maketitle

\begin{abstract}
アメリカ議会における立法生産性の議論は、議会が適切に機能しているのかという問題関心から多くの分析が蓄積されてきた。しかし、立法生産がアメリカ社会に応答的な形でなされてきたのかについては十分な検討がなされていない。本稿では、1994年から2013年までのデータを用いて、政策領域に対する社会的な注目度が、当該政策領域における立法生産性といかなる関係を示すのかを分析した。分析の結果、政策領域に対する注目度が高まることで、議会の生産性が低下し得ることが示された。加えて、こうした影響は議会の審議段階によっても異なることが観察された。
\end{abstract}


\section{はじめに}
- motivation: 
%アメリカ議会における立法生産性の議論は、議会が適切に機能しているのかという問題関心から多くの分析が蓄積されてきた。
\\
 アメリカ議会が十分に機能しているのかという問題は、議会に対する根本的な課題のひとつとして大きな関心の対象となってきた。フィリバスター\citep*{Koger2010-uc,Binder2001-eb}や分割政府の増加\citet*{Fiorina1992-iz,Fiorina1992-mp}などによって立法の停滞状況が生じてきていることは多くの議会研究者に共有された問題意識であり、さまざまな視座から分析が蓄積されてきている。\\
- main contribution: 
%しかし、立法生産がアメリカ社会に応答的な形でなされてきたのかについては十分な検討がなされていない。本稿では、1984年から2013年までのデータを用いて、政策領域に対する社会的な注目度が、当該政策領域における立法生産性といかなる関係を示すのかを分析した。
\\
 本稿では、議会が社会に応答しているのかという問題を立法生産性の観点から検討し、社会的に関心の高まっている政策領域では、逆に立法が停滞するという関係が観察されうることを指摘する。近年、アメリカ議会の機能をめぐる議論の中で、社会的な関心が高い政策領域における立法活動をどの程度行えているのかが重視されるようになってきている。\citep*{Binder2003-bn,Binder2017-wr,Adler2013-ay}しかし、こうした議論が前提としている社会的な注目度が高まれば法案審議が加速するという関係は、有権者全体に共有される選好という実証的に正当化困難な仮定に支えられている。本稿は、法案審議と社会的注目度のデータを用いて注目度の高い政策領域で審議が停滞するという傾向が存在することを示す。\\
 近年の立法生産性(legislative productivity)にかかわる議論はDavid Mayhewの\textit{Divided We Govern}を転機として様々な議論が蓄積されてきた。\citet*{Mayhew1991-rq}が分割政府と立法生産性との間に有意な関係が観察できないと指摘したことを皮切りに、歴史的な変化\citep*{Fiorina1992-iz}や議員の選好による説明\citep*{Krehbiel1998-ob,Krehbiel2010-ob}、政党と分極化\citep*{Thorson1998-vs,Jones2001-ds}、社会的関心度合いとの関係\citep*{Gibson1995-ou,Coleman1999-ld,Binder1999-fl,Binder2003-bn,Binder2017-wr}など様々な面から検討が重ねられてきた。\\
 Sarah Binderは立法生産性の議論に、「議会が社会の問題に適切に対応しているのか」という応答性の問題から再検討を加え、社会的な注目度が議会における膠着状態を緩和させる可能性を指摘した。\citep*{Binder2003-bn}\\
\\
- in detail discription:
%分析の結果、政策領域に対する注目度が高まることで、議会の生産性が低下し得ることが示された。加えて、こうした影響は議会の審議段階によっても異なることが観察された。
\\
 本稿は\citet*{Adler2013-ay}による分析における
議会の立法生産性と社会的注目度との関係は、その重要性に比して分析が少ない\footnote{重要な例外として\citet*{Adler2013-ay}}。本稿では政策領域への注目度と立法生産との関係を分析することで、その重要性を確認すると共に従来の議論で論じられてこなかった関係性の存在を指摘する。従来も社会的な関心度合いの高さが議会における立法生産性に対していかなる影響を及ぼすかについては議論がなされてきた。しかし、それらの\\
 また、先行研究では議会審議の程度を論じる際に、異なる法案審議の段階を用いた分析がなされてきた。本稿では、異なる審議段階における議会の応答性を検討することで、社会的な注目度が法案審議に対して与える影響は、どの段階の法案審議かによっても異なっていることを指摘する。\\

\section{先行研究}
\subsection{立法生産性の説明モデル}
 議会研究において、立法生産性はどのように論じられてきたのだろうか。以下では、立法生産性をめぐる近年の議論を概観し、政策領域ごとに立法生産を検討することの重要性と、そうした分析が十分なされていないことを指摘する。\\
 議会の立法生産性についての議論は、David Mayhewの\textit{Divided We Govern}を転機として、近年議論が蓄積されてきた。\citet*{Mayhew1991-rq}は、従来比較的広く論じられてきた命題である、「議会多数派と大統領の選出政党が異なる分割政府のもとでは、議会での法案審議は停滞する」という命題に対して、重要な法案\footnote{アメリカ議会における立法では、多数の法案が提出される一方で、実際に審議される法案は極めて少ない。提出される法案の中で、委員会での審議にかけられるものは例年提出法案の1割程度に止まっている。Mayhewは一定の基準を満たした法案を対象として分析を構成することで従来指摘されていた関係が実際には常に観察されるものではないことを示した。}についての議決数は分割政府下でも明確に低下しているわけではないことを示した。\\
 立法生産性と分割政府との関係はその後も多面的に検討がなされた。\citet*{Krehbiel1998-ob,Krehbiel2010-ob}は議会と大統領の選出政党の組み合わせではなく、議員個々人の政策的立場とその組み合わせから立法生産性を論じた。\\
 Sarah Binderは従来の立法生産性の議論が、単に議会が法案をどの程度審議してきたのかという数の問題に終始してきたことを問題視した。むしろ重要なのは社会的な課題を議会がどの程度解決できているかという点であり、課題のリストの中から議会がどの程度それらの課題に対して、法案提出という点で応答しているのかが重要であると論じた。\citep*{Binder2003-bn,Binder2017-wr}\\
 \citet*{Binder2003-bn}の議論の主眼は、立法のこう着状態を議会間の政策位置の違いなどに注目して説明する点にあり、政策領域への注目度の高さがこう着状態を緩和する傾向にあることを示唆しているに過ぎない。この点をさらに検討しているのが\citet*{Adler2013-ay}である。この議論では、政策領域への注目度が当該政策領域における立法生産を促進すると指摘している。\\
 政策領域ごとの検討を重視する議論は近年の議会をめぐる議論の中で注目されている視座である。\citet*{Lapinski2008-lr,Lapinski2013-jl}はアメリカ史の知見を基盤とした政策領域の分類方式を作成し、分極化や立法生産といった近年のアメリカ政治で注目されている課題は、政策領域ごとに異なる振る舞いを示すことを指摘している。\\

\subsection{「政策領域への注目度」とは何か}
 \citet*{Binder2003-bn}や\citet*{Adler2013-ay}の議論は、従来の議会におけるこう着状態をめぐる議論が注目してきた政治的変数だけではなく、社会的変数である有権者の関心や社会的な注目度といった要素に目を向けることの重要性を示すものである。では、そうした社会的な注目は従来どのように論じられてきたのだろうか。以下では、政策領域への社会的な注目度を論じてきた先行研究を検討し、注目度という変数が議会審議を促進する要因として理解されてきたことを確認する。\\
 政策領域に対する注目度を説明変数として政治現象を論じる枠組みは大まかには2つに分類できる。第一には、議員が選挙区の関心をいかに反映しているのかを主眼に据えた分析である。第二に挙げられるのは、社会的な関心の高さを、政策変化の契機として理解する議論である。\\
 第一の、議員と選挙区との関係を論じる議論では、議員は選挙区の関心の高さに応じて法案の提出、投票を行なっていると考える。\\
 第二の、政策変化の契機として注目度の高さを捉える分析では、特定の政策領域が関心を集めることで、従来の政治的均衡が崩壊し、大きな政策的変化につながると考える。\citep*{Baumgartner2015-ee,John2018-im}\citet*{Baumgartner1993-bc,Baumgartner2015-ee}は、政策の多くが極めて変化しにくいものである一方で、一定の条件下では急速かつ大規模な変化が生じていることを示した。こうした一定期間の強い安定性と短期にかつ急速な変化という両面の現象を説明するために、検討されたのが社会的な注目という変数である。\\
 こうした社会的注目によって政策的変化がもたらされるという発想はその後の政策変化の議論に引き継がれている。\citet*{Birkland1997-lq,Birkland1998-xp}は、自然災害などの社会的に注目を集める現象が生じた場合に、従来安定性が高かった政策状況が変化しうることを指摘した。\citet*{Sabatier1993-id}の議論はこうした社会的関心の高さという要因が、政治的アクターの連合状況を媒介して政策変化を生じうることを指摘している。\\
 以上の議論は共に、政策領域に対する注目度や関心の高さを、議会審議や政策変化をもたらす要因として理解している。こうした社会的な注目度が政策変化や積極的な議員行動をもたらすという発想は、議会研究においても引き継がれている。\citet*{Binder2017-wr}や\citet*{Adler2013-ay}の注目度がこう着状態を緩和するという議論だけではなく、議会における政党の役割を論じた\citet*[Ch.6]{Cox2005-pn}においても、政党が自らに不都合な法案審議を阻害するコストとして社会的な注目度が論じられている。これも、注目度が高いという要因が政策的変化を促進する機能を果たしているという理解である。\\

\subsection{注目度による負の影響}
 他方で、政策領域に対する注目度は必ずしも政策変化や審議を推し進める変数として機能するとは限らない。以下、社会的な注目度の高さが政策変化を阻害する可能性について検討する。\\
 政策領域が社会的な注目を集めるという現象は、法案を提出する側からすれば自らの法案に対する追い風として機能することが考えられる。議員が選挙区民の代表者である以上は、有権者の意思を無視して立法活動を行うということは選挙上のコストを支払うことになる。もし、社会的に注目度の高い法案に対して反対するという行為が有権者の意思にそぐわないものであれば、反対の立場を表明した議員は選挙戦略上一定のコストを支払うことになるだろう。こうして、社会的な注目度の高い法案に反対する、もしくはその審議を阻害することは選挙におけるコストとして機能し、結果として法案審議がより容易に進む要因となると考えられる。これが先行研究が社会的な注目度を法案通過や政策変化にとって好ましい状況として理解してきた所以である。\\
 しかし、この議論には重要な仮定が存在している。社会的に注目を集める法案に対する有権者もしくは選挙区民の立場が提出された法案を可決すべきというものである場合に、以上の議論は成立する。端的に言えば、法案に対して賛成することが、多くの選挙区民にとって好ましいのであれば社会的注目度はこう着状態を緩和すると考えられる。\\
 では、有権者の多くが特定の政策的立場を共有しているという現象はアメリカ政治でどの程度生じているといえるだろうか。エリートレベルで考える場合、結論は明確だろう。アメリカ史の中で党派的な投票行動が緩和されるのは稀な現象であり、多くの時期は党派的な対立が存在してきた。\citep*{Poole2007-ir,}現代に至っては、政治的対立のない政策領域を探す方がむしろ困難ともいえる。端的に言えば、「議員の大多数にとって反対することにより選挙上のコストを支払うことになる政策」が存在するために必要な、有権者一般における広範な合意という前提条件は非現実的な仮定であるということである。\\
 アメリカ社会において広範な合意を形成することが困難であるということは、何らかの立法提案に対して、反対することで選挙上の利益を得ることのできる議員が一定数存在することを意味する。例えば、連邦議会で人工妊娠中絶の権利を立法化しようとする動きは、リベラルな州において一定の評価を受けると考えられる。この場合、リベラルな選挙区に基盤を持つ議員は法案に賛成し、議会を通過するための努力することで選挙において自身の実績を主張することができる。では、保守的な基盤を有する議員にとってはこの法案に対していかなる応答が適切だろうか。保守的な有権者にとって、少なくとも連邦レベルにおいて人工妊娠中絶の権利が法制化されることは避けたい事態だと考えられる。この場合、保守的な議員にとって最適な行動は、この法案の成立を阻止することで「望ましくない法案を阻止した」という実績を得るということになるだろう。\\
 では、連邦議会において立法提案に反対するだけでなく、その立法を阻害することは実際に可能なのだろうか。誰が法案を提出するのかにも依存するが、最も困難な場合においても法案を促進することは可能であると考えられる。提出された法案を廃案に追い込む際、最も困難なのは、多数党の提出した法案を少数党が阻害するという場合だろう。事実、多数党は議場において法案審議を促進、阻害するための制度的権限を多数有している。\citep*{Aldrich1995-xf,Cox2005-pn,Cox2007-xq,Sinclair2016-kh}多数党は委員長職を独占することで、提出された法案の中からどの法案を実質的な検討の俎上に載せるのか判断することが可能であり、全院委員会などの議場における手続きも多数党に有利な条件として機能している。他方で、少数党であっても法案審議に対して一定以上の影響を与えていることが近年指摘されている。\\
 すなわち、政策領域に対して関心が高まるという現象は、法案審議を停滞させる機能を果たす場合もあり得るということである。\\
 以下では、社会的な注目度と法案審議との関係性を観察データを用いて分析する。\\

\section{分析方法}
\subsection{データ}
 分析に際しては、Comparative Agendas Project(CAP)によって収集・整理されたデータセットを利用する。\\

\subsubsection{従属変数}
 本稿の従属変数は、政策領域ごとの立法生産性である。実際の立法生産性の計測方法は様々にあり得るが、本分析では政策領域ごとに審議された法案の数を採用する。\\
 CAPのデータセットには、議会に提出された法案をCAPの用いる政策領域ごとに分類したデータが公開されている。本分析では法案が提出された年に政策領域ごとの法案数を合算した値を従属変数として採用する。\\
 加えて、法案審議に見られる様々な傾向は法案の審議段階によっても異なると考えられる。アメリカ連邦議会には、例年数千件から一万件近い数の法案が提出されている。他方、決議が行われる法案はその内の1割程度である。最終的は両院を通過した法案に大統領が署名を行なってはじめて法案が法律となる。したがって、本稿では審議の各段階、具体的には\textcircled{\scriptsize 1}法案提出、\textcircled{\scriptsize 2}議決、\textcircled{\scriptsize 3}立法化の3段階についてそれぞれ政策領域ごとの法案数\footnote{本稿では法案数をBillとJoint Resolutionの合算値によって導出している。したがって、Simple ResolutinoとConccurent Resolutionは分析から除外されている。これは、立法化され得る議案が前者2種類のみであり、後者は制度上立法化できないためである。\citep*{US_Senate2020-xq}この操作化については、\citet*{Adler2013-ay}も同様の処理を行っっており、妥当であると思われる。}を集計し、それぞれを従属変数とする分析を行った。\\

\subsubsection{独立変数}
 本稿の独立変数は、政策領域に対する注目度である。本分析では、\citet*{Adler2013-ay}の議論を参考に2つの変数を用いて注目度を計測した。\\
\paragraph*{メディア報道量}
 第一に、メディアの報道量である。具体的にはNew York Timesの報道について、CAPが分類した政策領域ごとの報道量を計算している。\\
 本分析では、メディア報道量の計算式を修正して用いている。これは、CAPの提示する計測方式では異なる年度間の比較を行う上で不適切であると考えられるからである。CAPによるメディア報道量の計測方法は、New York Times Indexの奇数ページに記載された最初の記事をCAPの分類コードに従って分類している。コーディングされた各記事は、記事に付されている概要文の長さによって、長・中・短の3つに割り付けられており、単なる記事の本数ではなく、その重要性を加味することができる。\\
 このサンプリング方法は、2つの面で複数年度の比較に適さない特性を有する。第一に、New York Times Indexは年度によってその全ページ数が異なるため、各記事の長さについての情報は異なる年度で単純な比較ができない。第二に、年度毎にページレイアウトが若干異なるため、ページ毎の記事数も年度によって異なっている。\\
 年度ごとの政策領域毎の報道量を説明変数として用いる場合、最も簡単な操作化は、年度ごとの各政策領域における報道量を、記事の重みを加味して合算することである。この場合、政策領域$i$における報道量$MediaCoverage_i$は、記事の長さ$j = 1,2,3$と長さ$j$の記事数$entry_{ij}$によって表すことができる。\\
\begin{align}
  MediaCoverage_i =  \sum_{j=1}^3 entry_{ij}*j
\end{align}
しかし、上述した通り年度毎にNew York Times Indexのページ数やページ毎の記事数が異なるため、本来であれば、この両者を加味した形で操作化することが望ましい。したがって、本稿では、(1)式を年度$k$の全ページ数$NumOfPage_k$とページ毎記事数$PerPage_k$によって調整した値を報道量の計算んしきとして採用する。すなわち年度$k$における政策領域$i$の報道量は以下の式で表される。\\
\begin{align}
  MediaCoverage_{ik} = \frac{PerPage_k}{NumOfPage_k} \sum_{j=1}^3 entry_{ij}*j
\end{align}
端的に言えば本式は、(1)で導出される各記事の「長さ」を、その年のNew York Times Indexのページ数が多い場合にはより短く、その年の各ページに記載された記事が多い場合にはより長く評価した値である。

\paragraph*{有権者の関心度合い}
 政策領域に対する社会的な関心の度合いを示す第二の独立変数は有権者が各政策領域に対してどの程度関心を払っているのかという問題である。Gallup社は「今日この国が直面している最も重要な問題はなにか("the most important problem facing this country today.")」という質問を定期的に有権者に対し行っている。\citep*{Gallup_Inc2006-vs}\\
 CAPはこのGallupによる定期的なサーベイの質問票を利用し、CAPが用意している政策領域それぞれについて有権者がどの程度関心を寄せているのかを年度ごとのデータとして用意している。本分析では、Gallup MIPで各政策領域が最も重要な争点であると答えた有権者の比率を有権者の各政策領域に対する関心度合いを示す指標として用いる。\\

\subsubsection{統制変数}
 法案審議がどの程度行われるのかは、必ずしも社会的な要因によってのみ影響されるわけではない。特に独立変数との関係で統制すべきと判断される要素については統制変数として回帰式に投入した。\\
 第一に、分割政府ダミーである。分割政府と統一政府とでは、立法生産に対して異なる状況が観察されることが指摘されてきた。\\
 第二に、選挙年ダミーである。本分析では年度毎に提出された法案について集計した上で分析を行っている。中間選挙を含めた選挙年と非選挙年では提出される法案数に大きな乖離が存在する。そのため、連邦レベルの選挙が存在する年とそうでない年を区分するために選挙年ダミーを投入した。\\
 第三に、政策領域ダミーを投入し、政策領域ごとの扱われやすさを統制している。例えば、マクロ経済と運輸とでは例年提出される法案数も有権者やメディアでの報道数にも差がある。政策領域を統制することで、これら所与の差異による影響を低減させる必要がある。\\
 第四に、多数党ダミーを投入している。多数党が法案審議過程に対して影響力を行使していることはしばしば指摘される事実である。\citep*{Rohde1991-da,Aldrich1995-xf,Cox2005-pn,Cox2007-xq}したがって、法案審議の過程においても多数党によって提出された法案と、少数党によって提出された法案では異なる取り扱いがなされると考えることが妥当である。この点についてもダミー変数を投入して統制している。\\
\subsection{分析モデル}
 分析には単純な最小二乗法(OLS)を利用し、政策領域ごとの\textcircled{\scriptsize 1}法案提出数(上院/下院)、\textcircled{\scriptsize 2}議決に付された法案数(上院/下院)、\textcircled{\scriptsize 3}立法化された法案数を従属変数としたモデルについて上述の独立変数・統制変数を投入した分析を行った。\\

\section{分析結果}
\begin{table}[h]
  \caption{House (Bill Introduction)}
  \centering
  \hspace{-1cm}

  \begin{tabular}[t]{lccc}
    \toprule
      & NYT v. Bill & MIP v. Bill & NYT + MIP v. Bill\\
    \midrule
    NYT Coverage & \num{0.862} &  & \num{0.026}\\
    & (\num{0.701}) &  & (\num{0.991})\\
    Gallup MIP &  & \num{85.433}* & \num{85.371}*\\
    &  & (\num{0.015}) & (\num{0.016})\\
    Divided Government (dummy) & \num{-1.972} & \num{-1.832} & \num{-1.836}\\
    & (\num{0.567}) & (\num{0.592}) & (\num{0.593})\\
    Bills from Majority Party (dummy) & \num{22.324}*** & \num{22.324}*** & \num{22.324}***\\
    & (\num{0.000}) & (\num{0.000}) & \vphantom{1} (\num{0.000})\\
    Major Topic & Yes & Yes & Yes\\
    Election Year (dummy) & \num{-43.996}*** & \num{-44.021}*** & \num{-44.021}***\\
    & (\num{0.000}) & (\num{0.000}) & (\num{0.000})\\
    Num.Obs. & \num{841} & \num{841} & \num{841}\\
    \midrule
    AIC & \num{8878.7} & \num{8872.7} & \num{8874.7}\\
    BIC & \num{9001.8} & \num{8995.8} & \num{9002.5}\\
    Log.Lik. & \num{-4413.343} & \num{-4410.345} & \num{-4410.345}\\
    F & \num{36.541} & \num{37.045} & \num{35.520}\\
    RMSE & \num{46.71} & \num{46.54} & \num{46.57}\\
    \bottomrule
    \multicolumn{4}{l}{\rule{0pt}{1em}. p $<$ 0.1, * p $<$ 0.05, ** p $<$ 0.01, *** p $<$ 0.001}\\
  \end{tabular}
  \end{table}

\begin{table}[h]

  \caption{Senate (Bill Introduction)}
  \centering
  \begin{tabular}[t]{lccc}
  \toprule
    & NYT v. Bill & MIP v. Bill & NYT + MIP v. Bill\\
  \midrule
  NYT Coverage & \num{-0.860} &  & \num{-1.382}\\
   & (\num{0.660}) &  & (\num{0.484})\\
  Gallup MIP &  & \num{49.936} & \num{53.232}.\\
   &  & (\num{0.101}) & (\num{0.084})\\
  Divided Government (dummy) & \num{-4.935}. & \num{-5.058}. & \num{-4.850}\\
   & (\num{0.100}) & (\num{0.090}) & (\num{0.105})\\
  Bills from Majority Party (dummy) & \num{18.918}*** & \num{18.919}*** & \num{18.918}***\\
   & (\num{0.000}) & (\num{0.000}) & \vphantom{1} (\num{0.000})\\
  Major Topic & Yes & Yes & Yes\\
  Election Year (dummy) & \num{-20.686}*** & \num{-20.697}*** & \num{-20.702}***\\
   & (\num{0.000}) & (\num{0.000}) & (\num{0.000})\\
  Num.Obs. & \num{839} & \num{839} & \num{839}\\
  \midrule
  AIC & \num{8622.2} & \num{8619.6} & \num{8621.1}\\
  BIC & \num{8745.3} & \num{8742.7} & \num{8748.9}\\
  Log.Lik. & \num{-4285.111} & \num{-4283.824} & \num{-4283.571}\\
  F & \num{17.323} & \num{17.480} & \num{16.790}\\
  RMSE & \num{40.59} & \num{40.53} & \num{40.54}\\
  \bottomrule
  \multicolumn{4}{l}{\rule{0pt}{1em}. p $<$ 0.1, * p $<$ 0.05, ** p $<$ 0.01, *** p $<$ 0.001}\\
  \end{tabular}
  \end{table}

\begin{table}[h]

  \caption{House (Bill On Passage Vote)}
  \centering
  \begin{tabular}[t]{lccc}
  \toprule
    & NYT v. Bill & MIP v. Bill & NYT + MIP v. Bill\\
  \midrule
  NYT Coverage & \num{-0.343}* &  & \num{-0.378}**\\
   & (\num{0.015}) &  & (\num{0.008})\\
  Gallup MIP &  & \num{2.741} & \num{3.637}\\
   &  & (\num{0.214}) & (\num{0.102})\\
  Divided Government (dummy) & \num{0.394}. & \num{0.342} & \num{0.400}.\\
   & (\num{0.071}) & (\num{0.117}) & (\num{0.067})\\
  Bills from Majority Party (dummy) & \num{4.273}*** & \num{4.276}*** & \num{4.275}***\\
   & (\num{0.000}) & (\num{0.000}) & \vphantom{1} (\num{0.000})\\
  Major Topic & Yes & Yes & Yes\\
  Election Year (dummy) & \num{-1.642}*** & \num{-1.641}*** & \num{-1.643}***\\
   & (\num{0.000}) & (\num{0.000}) & (\num{0.000})\\
  Num.Obs. & \num{819} & \num{819} & \num{819}\\
  \midrule
  AIC & \num{4106.5} & \num{4111.0} & \num{4105.7}\\
  BIC & \num{4224.2} & \num{4228.7} & \num{4228.1}\\
  Log.Lik. & \num{-2028.234} & \num{-2030.494} & \num{-2026.852}\\
  F & \num{40.826} & \num{40.411} & \num{39.319}\\
  RMSE & \num{2.92} & \num{2.93} & \num{2.92}\\
  \bottomrule
  \multicolumn{4}{l}{\rule{0pt}{1em}. p $<$ 0.1, * p $<$ 0.05, ** p $<$ 0.01, *** p $<$ 0.001}\\
  \end{tabular}
  \end{table}

\begin{table}[h]

  \caption{Senate (Bill On Passage Vote)}
  \centering
  \begin{tabular}[t]{lccc}
  \toprule
    & NYT v. Bill & MIP v. Bill & NYT + MIP v. Bill\\
  \midrule
  NYT Coverage & \num{-0.361} &  & \num{-0.379}\\
   & (\num{0.255}) &  & (\num{0.237})\\
  Gallup MIP &  & \num{0.946} & \num{1.848}\\
   &  & (\num{0.848}) & (\num{0.712})\\
  Divided Government (dummy) & \num{-1.073}* & \num{-1.128}* & \num{-1.071}*\\
   & (\num{0.028}) & (\num{0.020}) & (\num{0.029})\\
  Bills from Majority Party (dummy) & \num{4.582}*** & \num{4.583}*** & \num{4.583}***\\
   & (\num{0.000}) & (\num{0.000}) & \vphantom{1} (\num{0.000})\\
  Major Topic & Yes & Yes & Yes\\
  Election Year (dummy) & \num{-3.926}*** & \num{-3.925}*** & \num{-3.926}***\\
   & (\num{0.000}) & (\num{0.000}) & (\num{0.000})\\
  Num.Obs. & \num{831} & \num{831} & \num{831}\\
  \midrule
  AIC & \num{5516.8} & \num{5518.1} & \num{5518.7}\\
  BIC & \num{5639.6} & \num{5640.9} & \num{5646.2}\\
  Log.Lik. & \num{-2732.417} & \num{-2733.067} & \num{-2732.347}\\
  F & \num{47.390} & \num{47.263} & \num{45.451}\\
  RMSE & \num{6.58} & \num{6.59} & \num{6.59}\\
  \bottomrule
  \multicolumn{4}{l}{\rule{0pt}{1em}. p $<$ 0.1, * p $<$ 0.05, ** p $<$ 0.01, *** p $<$ 0.001}\\
  \end{tabular}
  \end{table}

\begin{table}

  \caption{Bill Enacted}
  \centering
  \begin{tabular}[t]{lccc}
  \toprule
    & NYT v. Bill & MIP v. Bill & NYT + MIP v. Bill\\
  \midrule
  NYT Coverage & \num{-0.282} &  & \num{-0.290}\\
    & (\num{0.293}) &  & (\num{0.287})\\
  Gallup MIP &  & \num{0.052} & \num{0.741}\\
    &  & (\num{0.990}) & (\num{0.862})\\
  Divided Government (dummy) & \num{-0.652} & \num{-0.695}. & \num{-0.651}\\
    & (\num{0.116}) & (\num{0.093}) & (\num{0.117})\\
  Bills from Majority Party (dummy) & \num{5.039}*** & \num{5.040}*** & \num{5.039}***\\
    & (\num{0.000}) & (\num{0.000}) & \vphantom{1} (\num{0.000})\\
  Major Topic & Yes & Yes & Yes\\
  Election Year (dummy) & \num{-3.156}*** & \num{-3.155}*** & \num{-3.156}***\\
    & (\num{0.000}) & (\num{0.000}) & (\num{0.000})\\
  Num.Obs. & \num{829} & \num{829} & \num{829}\\
  \midrule
  AIC & \num{5230.7} & \num{5231.8} & \num{5232.7}\\
  BIC & \num{5353.4} & \num{5354.6} & \num{5360.1}\\
  Log.Lik. & \num{-2589.345} & \num{-2589.915} & \num{-2589.330}\\
  F & \num{56.924} & \num{56.800} & \num{54.582}\\
  RMSE & \num{5.58} & \num{5.59} & \num{5.59}\\
  \bottomrule
  \multicolumn{4}{l}{\rule{0pt}{1em}. p $<$ 0.1, * p $<$ 0.05, ** p $<$ 0.01, *** p $<$ 0.001}\\
  \end{tabular}
  \end{table}

\subsection{法案提出における傾向}
 第一に法案提出について検討する。表1は、政策領域ごとに下院に提出された法案数を独立変数と統制変数を投入して回帰式を推定した点推定値及び各推定値のp値である。まず指摘できるのは、下院に提出された法案では有権者の関心度合い(Gallup MIP)が一貫して有意に正の推定値を示している点である。これは、議会に対して提出される法案全体で見た場合に、有権者が関心を持っている争点が議会に反映されやすいことを示しており、議員が選挙区民に応答的に法案提出を行っているという既存の知見と整合的な結果である。\citep*{Page1983-bx,Monroe1998-ty,Burstein2003-vs}\\
 次に、メディアによる報道の効果であるが、点推定値は正であるもののp値が大きく有意でない。メディアによる報道量と下院における法案提出傾向との間に明確な関係が存在しているとは言えないと理解すべきだろう。メディア報道と有権者の関心度合いを共に独立変数として投入したモデルでも点推定値、p値共にに大きな変化はないため、両者共にある程度一貫した傾向が観察されているといえる。\\
 統制変数についてはおおよそ先行研究に整合的な推定値が導出されているといえる。分割政府ダミーの点推定値は負であるものの、有意ではない。これは、統一政府か否かが立法生産に対して明確な影響を及ぼさないとした\citet*{Mayhew1991-rq}の知見と整合的なものである。また、多数党によって提出される法案が少数党に比べて多いことも、多数党が議事運営に際して大きな影響力を維持していることから考えれば妥当なことであろう。\citep*{Sinclair2016-kh,Cox2005-pn,Cox2007-xq}\\
 表2は、同じく提出法案についての分析を上院にも同様に行った結果である。おおよその結果は下院における分析と同様であり、有権者の関心度合いは有意でないものの点推定値が正である。統制変数の分割政府ダミーや多数党ダミーは推定値は異なるが同一の方向で推定値が導出されている。\\
 下院と異なる結果が示されているのはメディア報道の影響である。両者ともに有意差ではないものの、下院では点推定値が正であったのに対して上院では推定値が負になっている。分析全体として観察数が小さいため明確な議論はできないが、2つの可能性が示唆される。第一に、上院と下院において法案提出の傾向は大きな差異がなく、単に観察数が少ないため点推定値が安定してないことが考えられる。第二の可能性として、上院と下院には法案提出傾向に差異があり、観察数が少ないためにp値が大きくなっている可能性がある\footnote{本分析からはこれ以上の知見を導出することはできないが、上院と下院で法案提出後の審議過程に差異がある\citep*{Sinclair2016-kh}ことからも、両者における議員の法案提出傾向に差がある可能性はあり、更なる検討が必要であると考えられる。}。\\
 以上の知見をまとめると、法案の提出傾向に対して、有権者の関心度合いは正の影響を有すると考えられる。他方で、メディア報道は法案提出に対して明瞭な影響を与えているとは言えず、上院と下院との差異も法案提出に対して明確な差が示されるほどの違いがあるとは言えない。\\

\subsection{議決法案における傾向}
 第二に、実質的な法案審議に対してメディア報道と有権者の関心度合いがいかなる影響を示しているのかを分析する。表3は下院及び上院において議場での投票にかけられた法案数を被説明変数として同様の分析を行ったものである。\\
 表3は下院について議決結果への効果を分析したものである。まず指摘できるのは、法案提出の場合と異なり、議決に至る法案数に対しては、メディア報道量の点推定値が負になっている。これは5\%水準で有意な結果であり、有権者の関心度合いを統制しても同様の結果が示されている。\\
 ここから、メディアにおける報道が下院において審議を阻害する要因として機能する傾向にあることを指摘できる。先行研究を整理する中でも触れた通り、メディア報道などにより計測される関心度合いの高さは従来、立法や政策変化を促進する要因として理解されてきた。他方で、本分析結果が示しているのは、メディアにおける報道とその政策領域における立法生産との間に負の関係性が観察できるということである。\\
 有権者の関心度合いを示す点推定値は有意ではないものの、提出法案数同様に正の影響を示しており、こちらもメディア報道を統制しても大きな変化はない。統制変数についても大きな法案提出と大きな差異は見られない。\\
 興味深いのは、法案提出傾向においては効果の方向性が一致していたメディア報道量と有権者の関心度合いとが異なる効果を示している点である。この推定値を積極的に解釈するのであれば、有権者の関心動向とメディアにおける報道の傾向とは異なる側面を有しており、有権者の関心が高い場合には立法を促進しうるものの、メディアで積極的に取り上げられるような場合には立法にとっての阻害要因として機能するということになるだろう。メディアにおける報道が社会的な課題を直接的に反映するものではなく、むしろ一定の条件を満たす課題に焦点を当てることが指摘されている\citep*{Boydstun2013-ph}ことからも、メディアにおける報道が有権者の関心度合いと異なる傾向を示す可能性は十分にあると思われる。\\
 表4は上院について同様の分析を行ったものである。独立変数の点推定値については下院と大きな差はないが、上院の場合メディア報道の効果は有意でない。また、統制変数の中で分割政府が負の点推定値となっており、下院と若干異なっている。\\
 ここまでの分析結果をまとめると以下のことが言える。まず、上下院で共にメディア報道の議決数に対する効果が負になっており、メディア報道量と議案審議数の間に負の関係性が存在することを示すといえる。これは先行研究が注目度という変数を法案審議や立法生産を促す変数として理解してきたこと\citep*{Binder2003-bn,Binder2017-wr,Adler2013-ay}とは明確に異なる結果である。この効果は有権者の関心度合いを統制した場合でも大きな変化がない。すなわち、メディア報道は特に法案の議決段階において立法生産をむしろ抑制する傾向が観察されたということである。\\

\subsection{立法化された法案における傾向}
 最後に、法律の立法数を被説明変数としたモデルについても検討する。表5が立法数と各変数との関係を示したものである。メディア報道・有権者の関心度合い共に有意差は観察されないものの、点推定値は表3及び4と同様の傾向を示している。\\



\section{含意}

\newpage
\bibliography{reference}
\end{document}