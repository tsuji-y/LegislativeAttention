\documentclass{article}
\usepackage[utf8]{inputenc}
\usepackage{booktabs,siunitx,tabularx, natbib}
\usepackage[dvipdfmx]{hyperref}
\usepackage{comment}

\bibliographystyle{jecon}
\bibpunct[:]{(}{)}{,}{a}{}{,}

\title{What makes Legislative Attention}
\author{Yuki TSUJIMURA}
\date{Date}

\begin{document}
\maketitle

\begin{abstract}
アメリカ議会における立法生産性の議論は、議会が適切に機能しているのかという問題関心から多くの分析が蓄積されてきた。しかし、立法生産がアメリカ社会に応答的な形でなされてきたのかについては十分な検討がなされていない。本稿では、1984年から2013年までのデータを用いて、政策領域に対する社会的な注目度が、当該政策領域における立法生産性といかなる関係を示すのかを分析した。分析の結果、政策領域に対する注目度が高まることで、議会の生産性が低下し得ることが示された。加えて、こうした影響は議会の審議段階によっても異なることが観察された。
\end{abstract}


\section{はじめに}
- motivation: 
%アメリカ議会における立法生産性の議論は、議会が適切に機能しているのかという問題関心から多くの分析が蓄積されてきた。
\\
 アメリカ議会が十分に機能しているのかという問題は、議会に対する根本的な課題のひとつとして大きな関心の対象となってきた。\\
- main contribution: 
%しかし、立法生産がアメリカ社会に応答的な形でなされてきたのかについては十分な検討がなされていない。本稿では、1984年から2013年までのデータを用いて、政策領域に対する社会的な注目度が、当該政策領域における立法生産性といかなる関係を示すのかを分析した。
\\
 本稿では、議会が社会に応答しているのかという問題を立法生産性の観点から検討し、社会的に関心の高まっている政策領域では、逆に立法が停滞するという関係が観察されることを指摘する。\\
 近年の立法生産性(legislative productivity)にかかわる議論はDavid Mayhewの\textit{Divided We Govern}を転機として様々な議論が蓄積されてきた。\\
 Sarah Binderは立法生産性の議論に、「議会が社会の問題に適切に対応しているのか」という応答性の問題から再検討を加え、社会的な注目度が議会における膠着状態を緩和させる可能性を指摘した。\citep*{Binder2003-bn}\\
\\
- in detail discription:
%分析の結果、政策領域に対する注目度が高まることで、議会の生産性が低下し得ることが示された。加えて、こうした影響は議会の審議段階によっても異なることが観察された。
\\
 議会の立法生産性と社会的注目度との関係は、その重要性に比して分析が少ない\footnote{重要な例外として\citet*{Adler2013-ay}}。本稿では、政策領域への注目度と立法生産との関係を長期のデータから検討することで、従来の分析では析出されてこなかった関係を示す。\\
 加えて、先行研究では議会審議の程度を論じる際に、異なる法案審議の段階を用いた分析がなされてきた。本稿では、異なる審議段階における議会の応答性を検討することで、社会的な注目度が法案審議に対して与える影響は、どの段階の法案審議かによっても異なっていることを指摘する。\\

\section{先行研究}
\subsection{立法生産性の説明モデル}
 議会の立法生産性についての議論は、David Mayhewの\textit{Divided We Govern}を転機として、議論が蓄積されてきた。\citet*{Mayhew1991-rq}は、従来比較的広く論じられてきた命題である、「議会多数派と大統領の選出政党が異なる分割政府のもとでは、議会での法案審議は停滞する」という命題に対して、重要な法案についての議決数は分割政府下でも明確に低下しているわけではないことを示した。\\
 立法生産性と分割政府との関係はその後も多面的に検討がなされた。\\
 Sarah Binderは従来の立法生産性の議論が、単に議会が法案をどの程度審議してきたのかという数の問題に終始してきたことを問題視した。むしろ重要なのは社会的な課題を議会がどの程度解決できているかという点であり、課題のリストの中から議会がどの程度それらの課題に対して、法案提出という点で応答しているのかが重要であると論じた。\citep*{Binder2003-bn,Binder2017-wr}\\
 \citet*{Binder2003-bn}の議論の主眼は、立法のこう着状態を議会間の政策位置の違いなどに注目して説明する点にあり、政策領域への注目度の高さがこう着状態を緩和する傾向にあることを示唆しているに過ぎない。この点をさらに検討しているのが\citet*{Adler2013-ay}である。この議論では、政策領域への注目度が当該政策領域における立法生産を促進すると指摘している。\\

\subsection{「政策領域への注目度」とは何か}
 \citet*{Binder2003-bn}や\citet*{Adler2013-ay}の議論は、従来の議会におけるこう着状態をめぐる議論が注目してきた政治的変数だけではなく、社会的変数である有権者の関心や社会的な注目度といった要素に目を向けることの重要性を示すものである。\\
 政策領域に対する注目度を説明変数として政治現象を論じる枠組みは大まかには2つの議論が存在する。第一には、議員が選挙区の関心をいかに反映しているのかを主眼に据えた分析である。第二に挙げられるのは、社会的な関心の高さを、政策変化の契機として理解する議論である。\\
 第一の、議員と選挙区との関係を論じる議論では、議員は選挙区の関心の高さに応じて法案の提出、投票を行なっていると考える。\\
 第二の、政策変化の契機として注目度の高さを捉える分析では、政策の固着性を前提として政策変化が生じるためには、従来検討されて来なかった新しい側面が導入されることによる政治的均衡の変化が必要だと考える。\citep*{Baumgartner1993-bc}\\
 以上の議論はいずれにしても、政策領域に対する注目度や関心の高さを、議会のこう着状態を打破する変数として機能することを前提としている。\\
 他方で、政策領域に対する注目度は必ずしも政策変化や審議を推し進める変数として機能するとは限らない。\\
 すなわち、政策領域に対して関心が高まるという現象は、法案審議を停滞させる機能を果たす場合もあり得るということである。\\
 以下では、社会的な注目度と法案審議との関係性を観察データを用いて分析する。\\

\section{分析方法}
\subsection{データ}
 分析に際しては、Comparative Agendas Project(CAP)によって収集・整理されたデータセットを利用する。\\
\subsubsection{従属変数}
 本稿の従属変数は、政策領域ごとの立法生産性である。実際の立法生産性の計測方法は様々にあり得るが、本分析では政策領域ごとに審議された法案の数を採用する。\\
 CAPのデータセットには、議会に提出された法案をCAPの用いる政策領域ごとに分類したデータが公開されている。本分析では法案が提出された年に政策領域ごとの法案数を合算した値を従属変数として採用する。\\
 加えて、法案審議に見られる様々な傾向は法案の審議段階によっても異なると考えられる。アメリカ連邦議会には、例年数千件から一万件近い数の法案が提出されている。他方、決議が行われる法案はその内の1割程度である。最終的は両院を通過した法案に大統領が署名を行なってはじめて法案が法律となる。したがって、本稿では審議の各段階、具体的には\textcircled{\scriptsize 1}法案提出、\textcircled{\scriptsize 2}議決、\textcircled{\scriptsize 3}立法化の3段階についてそれぞれ政策領域ごとの法案数を集計し、それぞれを従属変数とする分析を行った。\\
\subsubsection{独立変数}
 本稿の独立変数は、政策領域に対する注目度である。本分析では、\citet*{Adler2013-ay}の議論を参考に2つの変数を用いて注目度を計測した。\\
 第一に、メディアの報道量である。具体的にはNew York Timesの報道について、CAPが分類した政策領域ごとの報道量を計算している。\\
 本分析では、メディア報道量の計算式を修正して用いている。これは、CAPの提示する計測方式では異なる年度間の比較を行う上で不適切であると考えられるからである。CAPによるメディア報道量の計測方法は、New York Times Indexの
\subsubsection{統制変数}
\subsection{モデル}
\section{分析結果}

\section{含意}

\newpage
\bibliography{reference}
\end{document}