\newpage
\section{議会審議はいかに説明できるのか?}
\subsection{はじめに}
 本章では、議会研究における蓄積を概観しつつ、既存の研究が社会的な関心事と議会における審議内容との連関について十分検討してこなかったことを指摘する。加えて、政策領域への注目度と、政策変化との関係を論じてきた一連の議論を参照しながら、議会の審議過程をよりよく説明するためには社会的な注目度と議会審議とがいかに連動しているのかを分析する必要があることを指摘する。最後に、現代のアメリカ議会における審議過程の説明に社会的な注目度を導入するためには、社会的な注目度が向上することによる負の側面に対しても目を向ける必要があることを指摘する。\\
\subsection{議会研究の問題意識: なぜ法案審議を研究するのか}
 議会研究を大別すれば、その背景には大きく3つの問題意識が存在しているといえる。第1には、議員個人と選挙区民との関係性から議会の法案審議を論じるという枠組みである。\citep*{Fenno1977-se,Fenno2000-up}議員は選挙区民の代表者であるという視角から、議員の法案提出の傾向などについて分析がなされてきた。第2に、議会を政党間競争の舞台として捉える見方がある。\citep*{Rohde1991-da,Aldrich1995-xf,Cox2005-pn,Cox2007-xq,Poole2017-ir}この議論では、議会において多数党がいかなる権限を有していかなどといった問題が分析の対象となってきた。第3に、議会がどの程度の法案審議を行っているのかという機能の面からも議論が蓄積されてきた。\citep*{Mayhew1991-rq,Binder2003-bn,Binder2017-wr}この議論の背景には、分極化が進展する中で、なぜ議会が機能不全に陥るのかという問題意識が存在する。\\
 以下ではそれぞれの問題意識と実際にいかなる議論がなされてきたのかについてより詳しく検討する。\\

\subsubsection{選挙区の代表者としての議員と議会}
\subsubsection{政党間競争の舞台としての議会}
 議会における政党の役割は、明示的に法文上で政党指導部に付与されているものでは必ずしもなく、実際にアクターが行動する中で立ち現れているという部分が大きい。ゆえに、議会において政党が果たしている役割については、根源的な疑問も含めて多数の理論的立場が存在している。最も政党の役割を小さく考える立場として\citet*{Krehbiel1998-ob,Krehbiel2010-ob}に代表される、議員の政策選好上の立場のみから法案審議過程を論じる議論が存在する。この議論では政党の役割はほぼ捨象され、議会内で鍵となる票を握る議員の重要性が強調される。より議会内政党の役割を重視する政党政府論の系譜は、議会において多数党が議案を積極的に前に進める役割を果たす\citep*{Rohde1991-da,Aldrich1995-xf}といった議論や、多数党の不利に働く議案を排除している\citep*{Cox2005-pn,Cox2007-xq}といった議論がなされる。\\
 本節では、政党間競争のあり方を論じてきた政党政府論の系譜を概観することで、議会内政党が法案審議に対していかなる影響を与えうると考えられてきたのかを検討する。加えて、複数の理論に対する実証的検討についても触れる。\\
\paragraph*{条件付き政党政府論}
\paragraph*{カルテル政党論}
 議会審議を論じるもうひとつの立場として、カルテル政党論\citep*{Cox2005-pn,Cox2007-xq}が存在する。この議論は議会で多数派を占める政党が議会における審議手続きを通じて、多数党内の多数派にとって不利になる法案を排除していると考えるものである。\\
 \citet*{Cox2005-pn,Cox2007-xq}は議員と政党について6つの仮定を設定する。第1に、議員が再選を追求するアクターであること、第2に政党の評判が議員個人の再選と政党の多数派維持のために重要な要素であること、第3に政党の評判はその立法成果に依存していること、第4に立法は集合行為問題を生じさせること、第5に政党は中心化された権限を行使して議員個人の行動を統制していること、第6に政党が議員を統制する主たる手段はアジェンダを設定することにあると仮定される。\\
 カルテル政党論において重要なのは、第6の仮説である政党によるアジェンダの設定という問題である。アジェンダを設定する権力とは、議場においてどの法案が、どんな手続きのもとで審議されるかを決定する権力のことである。具体的には、委員長が付託された法案の審議を遅らせることや、議事運営委員会において特別規則(special rule)の設定を行うことが想定される。逆に、全ての議員が参加できるような、委員会審査省略動議(discharge petition)への署名などはアジェンダ設定の権限に含まれない。このような、限定されたアクターのみが関与できる手段によって、議事が統制されていると考えるのがカルテル政党論におけるアジェンダ設定である。\\
 この議事統制の中で、特に注目されるのが法案の排除を通じた消極的なアジェンダ設定(negative agenda control)である。これは、特定の法案が審議過程で有利になるように権力を行使するのではなく、議案がそもそも議場で審議されることを回避することを意味する。こうした議事統制は、多数党にとって不利な法案について、議論や点呼投票などの立法行為を進めることで、政党の評判を落とし、多数党が選挙において不利になることを回避するために行われるとされる。\\
 以上のようにカルテル政党論によれば、法案審議において重要な要素はいずれの政党が多数派を握っているかという問題であり、議会における法案審議は主に多数派による議事統制を軸に展開すると考えられる。

\paragraph*{理論的立場をめぐる実証的知見}
 ここまで議会における法案審議をめぐる議論は理論的に極めて多様な立場が存在することを確認した。その中で、理論的議論は主に多数党の役割をいかに見積もるのかという問題を軸に展開しており、政党を全く考慮せず議員の政策選好のみによって説明可能であるとする議論\citep*{Krehbiel1998-ob,Krehbiel2010-ob}から、多数党が法案の審議過程に対して影響することで議事を一定程度統制しているとする議論\citep*{Aldrich1995-xf,Cox2005-pn,Cox2007-xq}まで多様な立場が存在することを確認した。\\
 以下では議案審議をめぐるこれらの理論的立場について、いかなる実証的根拠が提示されてきたのかについて概観する。結論を先取りすれば、実証分析の結果は概ね多数党の議案統制に対する影響力を肯定するものであるが、指標化・理論の包含性において不明瞭な点が残されていることを指摘する。\\
 議事進行に関わる実証的知見は、分析対象としては州と連邦レベルの議会を対象として、手法として主に点呼投票の議決結果を用いて実証が行われてきた。点呼投票を用いた分析では、投票に掛けられた法案が多数党の多数派が賛成しているにも関わらず否決される割合(roll rate)を指標として、どの程度政党が影響力を行使しているのかが分析されてきた。\citep{Magleby2018-rc,Richman2020-al}\\
 点呼投票を中心とした観察データを用いた分析の知見は、概ね多数党が法案審議過程に対して影響力を行使していることを示しており\citep*{Cox2001-eu,Lawrence2006-ed,Cox2010-gb,Meagher2012-to,Clark2012-lk}、こうした政党の役割を無視して単に議員個人の分布の問題としてのみ議会審議を扱うことはできないと思われる\footnote{これらの議論は主に下院を対象として蓄積されているものの、上院においても多数党による議案統制がなされていることが指摘されている。\citep*{Gailmard2007-yj,Hartog2008-xs}}。\\
 また、カルテル政党論の主張するアジェンダ設定の権能は、一定の条件のもとで発揮されることも指摘されている。\citet*[Ch6]{Cox2005-pn}は政党のアジェンダ設定にはコストが存在し得ると論じているが、これは経験的にも裏付けられており、議案審議を多数派が統制することは選挙においてコストとなりうること\citep*{Richman2015-xo}や下院での特別規則についての投票では党内の同質性が多数党の議事統制能力を条件づけていること\citep*{Finocchiaro2008-zh}などが指摘されている。多数党による統制能力が一定の条件のもとで生じるとする議論は議案の通過をめぐる点呼投票以外を対象とした分析で顕著であり、多数党の議事統制能力は段階的であることが示唆される。\\
 その一方で、純粋多数派理論の枠組みが全く説明能力を持たないとも言い難い。重要法案に限定した場合に純粋多数派理論が経験的根拠を持つと指摘する議論\citep*{Gray2019-sv}や純粋多数派理論の前提として、カルテル政党による議案統制を組み込むことによって理論予測の妥当性が向上することを州レベルで実証した知見\citep*{Crosson2019-xb}も存在しており、必ずしも多数党の議案統制のみによって議会審議が経験的に説明できている訳ではない。\\
 以上、議会審議過程における多数党の役割についての実証的な分析を概観した。近年の知見が議会審議の提供する見取り図は、現代のアメリカ議会において、多数党は議事手続きを通じた影響力を行使し、一定程度法案を統制しているが、その程度は委員会での検討や点呼投票といった議会審議過程の段階と党内の凝集性によって条件づけられているというものであろう。加えて、点呼投票の段階では\citet*{Krehbiel1998-ob,Krehbiel2010-ob}の指摘するようなピボット間の幅によって議案の通過しやすさに差異が生じてくると説明することで、議会の審議過程の説明としては妥当性が向上する可能性が示唆されている。\\
 特に法案審議をめぐって多数党が果たしてきた役割については、その具体的内容についても解明が進んでおり、おおよそ確実なものと考えて良いだろう。\citet*{Rosenthal2008-xb,Peters2010-ve}は、Nancy Pelosiの下院議長としての戦略的行動を描き出すことで、政党指導部がいかなる形で法案審議に関与してきたかを示している。\citet*{Sinclair1997-jm,Sinclair2016-kh}が論じる新しい法律制定過程においても、法案審議を前に進める上で、政党指導部が重要なアクターとして位置付けられている。\\

\subsubsection{議会の機能不全をめぐる議論}


\subsection{議会研究の枠組み: 何によって分析するのか}
 これらの3つの問題意識は、さまざまな分析手法を用いて分析されてきた。大まかには2つの分析枠組みがあったといえる。第1には何らかのイデオロギー「イデオロギー尺度(ideology score)」と呼ばれる指標を用いる手法である。第2の枠組みは政策領域ごとに議論を組み立てていくという方策である。以下それぞれについて概観する。\\


\subsection{先行研究の課題}
\subsubsection{分極化と立法過程}
\subsubsection{「議員のイデオロギー分布」の限界}
 本論は議会の機能と社会への応答性に主たる関心を有する。これは、議会が実際にいかなる法案を審議し、立法しているのかという問題を抜きにして議会の代表性を論じることは不可能であると考えるからである。\\
 加えて、議会審議の内容は個別議員や政党が所与に有する政策的立場やその分布と権力関係によって決定されると考えるのではなく、有権者やメディアなどといった議会を取り巻くさまざまな社会的要因によって規定されるというモデルを採用する\footnote{これは政策選好が全く効果を持たないということではなく、政策選好よりも重要な要素が存在しているのではないかということである。}。従来の議会研究の多くは、議員個々人が政策選好を有しており、選挙などによって生じる政策選好分布の変化によってのみ既存の法律が変更されるというモデルによって議会の法案審議を論じている。\citep*{Poole2017-ir,Krehbiel1998-ob,Cox2005-pn,Cox2007-xq,Aldrich1995-xf}しかし、こうした所与の政策選好分布と権力構図との組み合わせによって議会の法案審議を論じることにはいくつかの重要な問題が存在すると思われる。\\
 第一に、法案の初出と法律化とのズレの問題である。議会に提出される法案の多くは、初めて議会に提出されるものではなく、すでに複数の議会期で提出されたことのある内容の法案が形を変えながら再度提出されている。すなわち、提出されても法律化される場合とそうでない場合とが存在するということである。政策選好が法案審議の主たる説明要因であるとすれば、複数回法案が提出される間に少なくない数の議員構成の変化が生じる必要がある。\\
 しかし、実際には法案が複数回提出される間に必ずしも一定以上の構成員の変化が生じているとは考えにくい。各議会期において一定数の法案が審議され、可決されているのであり、可決されたものとされなかったものを「現状打開点」\citep*{Tsebelis2009-hf}に特定の会期のみ入ったとは考えにくい。そもそも、同じ議会期にほとんど同内容の法案が複数回提出される場合すらある。政策選好によって議決結果が決定されるとすれば、これは荒唐無稽な行動と言わざるを得ない。\\
 こうした問題を政策選好によって解決するために必要となるのが、政策選好の変化である。\citet*{Poole2017-ir}が論じるように、個々の議員が有する政策選好の理想点は時期によって異なっており、既存の理想点推定の手法はこうした変化に何らかの形で応答しているといえる\footnote{\citet*{Poole2017-ir}は理想点の変化を線形にモデル化しているが、ベイズ推定による事前分布設定を行うIRTを利用することでより柔軟な推定が可能であるとも指摘されている。\citep*{Caughey2016-ef}}。政策選好が時期によって変化するという枠組みを採用することで、異なる時期に異なる政治的判断を下すという現象を説明可能になる。\\
 他方で、政策選好の変化という枠組みは政策選好によって議決や法案提出といった議会における審議パターンを説明する利点を損なう側面を有する。政策選好によって議会の審議過程を説明し、付加的に手続きや規則による制約、政党の権限などを条件付けするという分析枠組みは、議員個人の政策選好が一定程度固定的である場合に意味を持つ。「保守的な議員が多い場合に保守的な法案が通りやすい」「保守的な議員が多くとも委員長をリベラルな議員が占めている場合には保守的な法案が通りにくい」といった説明は、保守/リベラルという政策選好が一定程度頑健である場合には意味を持つ。しかし、保守/リベラルという政策選好が場合によって、また異なる状況下で大きく変化するとしたら、そうした変数によって審議過程を説明することにどのような意味があるだろうか。\\
 こうした可変的な政策選好を推定することは、予測の面では大きな意味を有する。推定に対して一定の揺らぎを読み込むことは、未知の投票に対して議会がどのような応答をし得るかという問題に回答するためには、その説明能力を向上させるという意味で重要な役割を有する。\\
 他方で、イデオロギー尺度による、もしくはそれに制度的機構や権力分布を条件付けた法案審議の説明モデルは結局のところ、根本的な課題を克服できていない。「なぜ法案は何度も提出され、タイミングによってその成否が異なるのか」という問題は、イデオロギー尺度を説明変数とするモデルでは回答しにくい。\\
\subsubsection{政策領域への回帰}

\subsection{「政策領域への注目度」という変数}
 本論の第一の主張は、「議会における法案審議を論じる上で、政策領域に対する社会的な注目度合いが重要な変数である」というものである。こうした議論は、イデオロギー尺度を主軸とする議論に比べれば数は少ないものの、一定以上の蓄積が存在する。\\
 本論が、既存の政策領域への注目度と法案審議との関係を論じた多くの研究と明確に異なるのは、「政策領域への注目度は法案審議に対して負の影響を与えうる」という点であり、これが本論の第二の主張である。以下では、政策領域への注目度と立法過程との関係を論じてきた3つの議論を概観した上で、政策領域が注目されることによる負の側面の重要性を指摘する。\\

\subsection{説明変数として社会と注目度の概念}
 政策領域への注目度を扱った分析として古典的なものは\citet*{Kingdon1984-oq}の「3つの流れモデル(Multiple Stream Framework)」\footnote{\citet*{Kingdon1984-oq}が\textit{Agendas, alternatives, and public policies}の中で論じた政策変化をもたらす過程の議論にKingdon本人は明確な呼称を与えていない。本邦では「政策の窓モデル」などの呼称が一般的であるが、ヨーロッパ諸国などの適用においては\citet*{Zahariadis2003-ck}の用いた"Multiple Stream Framework(MSF)"の呼称が一般的であり、レビュー論文や教科書などでもこの呼称が用いられている。また、研究史上も政策のライフサイクル論に対して複数の流れの複合体として政策変化を論じた点にKingdonによる議論の重要性があるとの判断から本論では「3つの流れモデル」と表記する。}であろう。\citet*{Kingdon1984-oq, Kingdon2013-ac}は、組織的意思決定のゴミ缶モデル\citep*{Cohen1972-ym}を応用した大統領府の分析を通じて政策変化をもたらす要因を論じた。「3つの流れモデル」は、政策変化をもたらす過程を「問題の流れ」「政策の流れ」「政治の流れ」という3つの異なる相互に独立した過程の混合物として捉え、これらが合流することによって政策変化が生じる契機がもたらされるとする。この契機を「政策の窓」と呼び、3つの流れを合流させて契機をもたらし、それを掴んで政策的な変化をもたらすためには、政策企業家(policy entrepreneur)が重要な役割を果たすと論じられる。\citep*{Kingdon1984-oq,Kingdon2013-ac}\\
 このようなMSFモデルは、政策変化を説明するモデルとして有効であるとされ、議院内閣制の国における政策過程を含めた広い適用対象を包含するモデルとして様々な政策過程の分析において応用されてきた。\citep*{Rawat2016-ew,Jones2016-lc}\\
 しかしながら、MSFモデルはその予測可能性の低さ\citep*{}や概念整理が不十分である\citep*{John2018-im}などの課題を有していた。MSFモデルは政策過程をめぐる極めて複雑な問題状況を整理する補助線を提供するという意味で極めて重要な貢献を果たした一方で、「3つの流れ」とされる概念間には十分な対応関係は存在せず、それぞれの流れが具体的にいかなる要素を内包しているのかも判然としない。MSFモデルを実際に用いた多くの分析において、概念や操作化に一貫性がない\citep*{Jones2016-lc}ことも、このモデルが原因と結果の関係を論じるというよりむしろ、問題状況の整理と認識のための枠組みという側面が強いことを示している。\\
 こうした課題に対応する議論として、JonesとBaumgartnerによる一連の研究\citep*{Baumgartner2010-rl,Baumgartner2015-ee, Baumgartner2009-eb,Jones2005-bp}は政策変化の過程において一定期間の安定性と短期の急激な変動という特徴が存在することを指摘し、そうした現象が生じる要因を分析している。\citet*{Jones2005-bp}は、政策が基本的に強い現状維持バイアスを有することを前提としつつ、変化する契機として、既存の問題群に新しい争点が加わることで従来の均衡が崩壊し、大きな政策的変化が生じると論じている。このようにして、一定期間の安定と突然の大きな変化をモデル化したのが断続均衡理論である。\\
 断続均衡理論は「なぜ政策は変化するのか」に対する回答を提示するものの、「特定のタイミングで、なぜ特定の政策のみが重要な争点となり、変化をもたらすのか」という問題に対して十分な回答を提示していない。端的に言えば、何が原因となって変化が生じるのかについてはなお議論の余地がある。\citep*{John2018-im}\\
 何が変化をもたらすのかについては、さまざまな要因が検討がなされてきた。自然災害などの焦点となる出来事によって特定の政策領域に注目が集まるとする議論\citep*{Birkland1997-lq,Birkland1998-xp}、外生的要因が政治的連合の変化\citep*{Sabatier1993-id}や有権者の関心の変化\citep*{Bertelli2013-zq}をもたらすことで特定の政策領域が注目されるとする議論などがある。また、知識の更新によって政治的状況の変化を説明する議論\citep*{Baumgartner2010-rl}などが存在している。\\
 いずれにしても、これらの政策変化を説明する試みは、事件や事故、画期的な研究成果などの何らかの出来事によって、変化が引き起こされるというモデルを想定している。こうした発想は政策変化を扱う研究のみならず、多くの政策史の研究も何らかの形で引き継いでいるものと思われる。\\

\subsubsection{政策変化をもたらす「注目」}

\subsubsection{「注目」された政策は変化するのか?}
 前節では、政策変化を扱った諸研究における議論を概観し、人々の関心を惹くことによって政策が変化するというモデルが多くの研究者によって論じられてきたことを確認した。本節では、こうした見方は十分に政治的現象、特に法案審議を説明できていないことを指摘する。\\
 「多くの人々が関心を持っている」ということが法案審議を進める上でプラスの影響を持つということは一見すると非常に直感的なように思われる。しかし、これは常に存在する関係ではなく、むしろ今現在目の前で展開しているアメリカ政治の現実からは非常に半直感的な結論である。\\
 ひとつの例として、ティーパーティ運動と予算政策について検討してみよう。\\